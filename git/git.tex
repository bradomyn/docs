

\title{Git Resume}
\date{\today}

\documentclass[12pt]{article}
\newcommand \tb[1]{\textbf{#1}} 
\newcommand \ti[1]{\textit{#1}} 

\begin{document}
\maketitle

\begin{abstract}
Resume of Git commands \ldots
\end{abstract}

\section*{Creating a project}

\tb{Configuration Repository}

Starting with a repo: \\
\ti{
 git config --global user.name "Your Name" \\
 git config --global user.email bradomyn@gmail.com \\
 git config --global color.ui true \\
 git add [file] \\
 git commit -m 'first commit' \\ 
 git remote add origin git@github.com:bradomyn/test.git \\
 git push -u origin master \\
}

\tb{Staging}

 If a file ready exists and we use "add" the changes are in the 
 staging area, waiting for commit. \\

 \ti{git reset HEAD <file>} to unstage

\tb{Viewing status and differences}
\noindent
\ti{ git log -n when n is the numer of log messages that I want to see \\
 git status \\
}

\ti{git diff} diferences between working tree \& stagin area.
\ti{git diff --cached} stagin area \& repository
\ti{git diff HEAD} working tree \& repository


\ti{git commit -a}

That tells Git to commit all files that it knows about that have 
changed.

\ti{git add -i}

displays a prompt

\tb{Commiting}

\noindent

\ti{ git add some-file \\
     git commit -m "changes to some-file" \\ }
stag a file o part of a file, and then commit

\ti{git commit -m "balab" -a}
commits the latest version from your working  tree.


\ti{git commit -m "changes to some-file" some-file}
commits only this file


Breaking one commit in several(take a look at the documentation)


\tb{Patch Mode}
 
Using \ti{git add -i}, inside of the menu number 5. or 
\ti{git add -p}

hunk is a change in a file.

\textbf{Tagging}

\ti{git tag NAME-TAG BRANCH}
 
to see all the tags \ti{git tag}


\tb{Merging}

\ti{Stright Merging}

git checkout BRANCH
git merge BRANCH-TO-MERGE

\ti{Squashing Commits}

takes all the history of one branch and compresses it into one commit in the other branch

\ti{Cherry-Picking Commits}

to merge only one commit between branches and don’t need to do a full merge

\ti{git cherry-pick 321d76f}

\textbf{Rebase}
Rebasing takes the changes from a branch and replays them on top of another branch

First we have to change to branch where we want to get the changes 

\ti{git checkout BRANCH}

\ti{git rebase BRANCH-TO-REBASE}

remove the branch

\textbf{Branch}
Create a branch
\ti{git branch NEW-BRANCH  BASED-BRANCH}

\ti{git checkout -b alternate master} short cut for creating moving into a
branch

Rename a branch -m BRANCH-NAME BRANCH-NEW-NAME

Remove a branch

git branch -d BRANCH-NAME

\tb{Undoing}

git commit -c -a --amend

\ti{Reverting}

\ti{git revert -n NUMBER-COMMIT}

\ti{git commit -m "revert NUMBER-COMMIT"}

without the -n the revert commits directly

\ti{Resetting}

git reset COMMIT-NAME

--soft when you want to stage all the previous commits but not commit them.

--hard It removes the commit from your repository and from your working tree. It’s the
equivalent of a delete button on your repository with no “undo.”

\tb{Archive}

git archive --format=tar \
            --prefix="xxx" \
            | gzip > xxx.tar.gz



\textbf{Merge vs Rebase}

http://gitguru.com/2009/02/03/rebase-v-merge-in-git/

\textbf{Exclude files from git}

A global .gitignore file can also be used by adding one to your 
global git config. For example, you might create the file 
$~/.gitignore_global$ and add some rules to it. To add this to your 
config, run: 

\ti{git config --global core.excludesfile $~/.gitignore_global$}


\begin{verbatim}
# Compiled source #
###################
*.com
*.class
*.dll
*.exe
*.o
*.so

# Packages #
############
# it's better to unpack these files and commit the raw source
# git has its own built in compression methods
*.7z
*.dmg
*.gz
*.iso
*.jar
*.rar
*.tar
*.zip

# Logs and databases #
######################
*.log
*.sql
*.sqlite

# OS generated files #
######################
.DS_Store*
ehthumbs.db
Icon?
Thumbs.db

# Vim #
#######
*.swp
\end{verbatim}


\end{document}
